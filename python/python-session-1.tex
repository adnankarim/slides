% Created 2015-01-21 Wed 16:18
\documentclass[bigger, presentation]{beamer}
\usepackage[latin1]{inputenc}
\usepackage[T1]{fontenc}
\usepackage{fixltx2e}
\usepackage{graphicx}
\usepackage{longtable}
\usepackage{float}
\usepackage{wrapfig}
\usepackage{soul}
\usepackage{textcomp}
\usepackage{marvosym}
\usepackage{wasysym}
\usepackage{latexsym}
\usepackage{amssymb}
\usepackage{hyperref}
\tolerance=1000
\usepackage{minted}
\usetheme{Frankfurt}
\usecolortheme[RGB={0,104,139}]{structure}%deepskyblue
\usefonttheme{serif}  % or try serif, structurebold, ...
\setbeamertemplate{navigation symbols}[horizontal]
\setbeamertemplate{caption}[numbered]
\useinnertheme{rounded}
\setbeamercovered{transparent}
\usepackage{pgfpages}
\pgfpagesuselayout{resize to}[physical paper width=8in, physical paper height=6in]
\usepackage{array}
\usepackage{graphicx}
\usepackage{hyperref}
\usepackage[english]{babel}
\usepackage{pxfonts}
\usepackage{listings}
\lstset{numbers=left,numbersep=6pt,numberstyle=\tiny,showstringspaces=false,aboveskip=-50pt,frame=leftline,keywordstyle=\color{black},commentstyle=\color{orange},stringstyle=\color{black},}
\date{today}
\subtitle{Python | Session 1}
\providecommand{\alert}[1]{\textbf{#1}}

\title{Python}
\author{Sachin}
\date{\today}
\hypersetup{
  pdfkeywords={org mode, emacs, latex, beamer, pdf},
  pdfsubject={my first presentation made in org mode},
  pdfcreator={Emacs Org-mode version 7.9.3f}}

\begin{document}

\maketitle

\section{Intro}
\label{sec-1}
\begin{frame}
\frametitle{Python}
\label{sec-1-1}


\begin{itemize}
\item High level programming language
\item Used in Scientific computing, Application development, etc.
\end{itemize}
\end{frame}
\begin{frame}
\frametitle{Why Python?}
\label{sec-1-2}


\begin{itemize}
\item Easy to learn, read and, modify
\item Easy to implement
\item Object Oriented
\end{itemize}
\end{frame}
\section{Basics}
\label{sec-2}
\begin{frame}[fragile]
\frametitle{Lists}
\label{sec-2-1}

   \emph{List manipulation}


\begin{minted}[]{python}
empty_list = []
list = [1, 4, 9, 2]             # List
print list                      # [1, 4, 9, 2]
list.append(8)                  # [1, 4, 9, 2, 8]
list.pop()                      # 8
print list                      # [1, 4, 9, 2]
\end{minted}
\end{frame}
\begin{frame}[fragile]
\frametitle{Lists}
\label{sec-2-2}

   \emph{List manipulation- access elements}


\begin{minted}[]{python}
org = ["gnu.org", "emacs.org", "hive.org"] # List
print org[0]                   # gnu.org
print org[2]                   # hive.org

# Last elements
print org[-1]                  # hive.org
print org[-2]                  # emacs.org
\end{minted}
\end{frame}
\begin{frame}[fragile]
\frametitle{Lists}
\label{sec-2-3}

   \emph{List manipulation- slicing}


\begin{minted}[]{python}
org = ["gnu.org", "emacs.org", "hive.org"] # List
print org[:1]  # ['gnu.org']
print org[:2]  # ['gnu.org', 'emacs.org']
print org[:3]  # ['gnu.org', 'emacs.org', 'hive.org']
\end{minted}
\end{frame}
\begin{frame}[fragile]
\frametitle{Lists}
\label{sec-2-4}

   \emph{List manipulation- slicing}


\begin{minted}[]{python}
list = [9, 6, 3, 1, 7, 5, 0, 4, 8]
list[0:3] # [9, 6, 3]
list[1:3] # [6, 3]
list[1:1] # []
list[5:7] # [5, 0]
list[4:5] # ?
\end{minted}
\end{frame}
\begin{frame}[fragile]
\frametitle{Lists}
\label{sec-2-5}

   \emph{List manipulation}


\begin{minted}[]{python}
list = [9, 6, 3]
# Add list
new_list = ["Gandolf", "Gollum", "Aragron"]
my_new_list = list + new_list
# [9, 6, 3, 'Gandolf', 'Gollum', 'Aragron']
len(my_new_list)                # 6
list_in_list = list.append(new_list)
# [9, 6, 3, ['Gandolf', 'Gollum', 'Aragron']]
list_in_list[3][2]              # 'Aragron'
# What if I want to print 'Gandolf'?
\end{minted}
\end{frame}
\begin{frame}[fragile]
\frametitle{List}
\label{sec-2-6}

   \emph{Run for-loop over list}


\begin{minted}[]{python}
new_list = ["Gandolf", "Gollum", "Aragron"]
for item in new_list:
    print item

# Gandolf
# Gollum
# Aragron
\end{minted}
\end{frame}
\begin{frame}[fragile]
\frametitle{Dictionary}
\label{sec-2-7}



\begin{minted}[]{python}
empty_dict = {}
status = {
    'stdout': 'Hello',
    'stderr': None,
    'exit': 0,
}

print status['exit'] # 0
print status['stdout'] # 'Hello'

print status.keys() # ['stdout', 'stderr', 'exit']
print status.values() # ['Hello', None, 0]
\end{minted}
\end{frame}
\begin{frame}[fragile]
\frametitle{Dictionary}
\label{sec-2-8}

   \emph{run for-loop over a dictionary}

\begin{minted}[]{python}
numbers = {
    'one': 1,
    'two': 2,
    'three': 3,
    'four': 4
}

for k, v in numbers.iteritems():
    print k,v
\end{minted}
\end{frame}
\begin{frame}[fragile]
\frametitle{Functions}
\label{sec-2-9}

   \emph{Define a function}


\begin{minted}[]{python}
# Function definition
def greet():
    """Greet user."""
    print "Hello "

# Call a function
greet()
\end{minted}
\end{frame}
\begin{frame}[fragile]
\frametitle{Functions}
\label{sec-2-10}

   \emph{function return a value}


\begin{minted}[]{python}
# Function definition
def greet():
    """Greet user."""
    # return a string
    return "Hello "

# Call a function
print greet()
\end{minted}
\end{frame}
\begin{frame}[fragile]
\frametitle{Functions}
\label{sec-2-11}

   \emph{Function with argument}


\begin{minted}[]{python}
# Function definition
def greet(username):
    """Greet user."""
    print "Hello ", username

# Call a function
name="Sachin"
greet(name)
\end{minted}
\end{frame}
\begin{frame}[fragile]
\frametitle{Functions}
\label{sec-2-12}

   \emph{Function with argument}


\begin{minted}[]{python}
# Function definition
def greet(username):
    """Greet user."""
    print "Hello %s" % username

# Call a function
name="Sachin"
greet(name)
\end{minted}
\end{frame}
\begin{frame}[fragile]
\frametitle{Functions}
\label{sec-2-13}

   \emph{lambda function}


\begin{minted}[]{python}
(lambda x: x > 2)(3)  # True
(lambda x: x > 2)(1)  # False
(lambda x: x+10)(45)  # 55
\end{minted}
\end{frame}
\begin{frame}[fragile]
\frametitle{string method}
\label{sec-2-14}

   \texttt{format}

\begin{minted}[]{python}
"1st arg: {0}, 2nd arg: {1}".format(47, 11)
# 1st arg: 47, 2nd arg: 11

"1st arg: {0:.2f}, 2nd arg: {1:.1f}".format(47.874,
                                            11.345)
# 1st arg: 47.87, 2nd arg: 11.3
\end{minted}
\end{frame}
\section{Class}
\label{sec-3}
\begin{frame}[fragile]
\frametitle{Simple class}
\label{sec-3-1}



\begin{minted}[]{python}
class Animal(object):
    """Animal class"""
    def walk(self):
        print "Walking.."

    def eat(self, food):
        print "Eating %s" % food

    def fight(self):
        print "Fighting.."

if __name__=='__main__':
    animal_obj = Animal() # instance
    animal_obj.fight() # Fighting..
    animal_obj.eat("flesh") # Eating flesh
\end{minted}
\end{frame}
\begin{frame}[fragile]
\frametitle{Simple class}
\label{sec-3-2}



\begin{minted}[]{python}
class Animal(object):
    """Animal class"""
    def walk(self):
        print "Walking.."

    def eat(self, food="flesh"):
        print "Eating %s" % food

    def fight(self):
        print "Fighting.."

if __name__=='__main__':
    animal_obj = Animal() # instance
    animal_obj.fight() # Fighting..
    animal_obj.eat() # Eating flesh
\end{minted}
\end{frame}
\begin{frame}[fragile]
\frametitle{Inherit a class}
\label{sec-3-3}

   \emph{Inherit Animal class}


\begin{minted}[]{python}
class Cat(Animal):
    """Animal category: Cat"""
    def drink(self):
        print "Drink Milk"

if __name__=='__main__':
    cat_obj = Cat()  # instance
    cat_obj.drink() # Drink Milk
    cat_obj.walk() # Walking
    cat_obj.eat("Biscuit") # Eating Biscuit
\end{minted}
\end{frame}
\begin{frame}[fragile]
\frametitle{Class constructor}
\label{sec-3-4}

   \emph{init}


\begin{minted}[]{python}
class Calculator():
    """
    A calculator with offset.
    """
    def __init__(self, offset=0):
        self.offset = offset

    def add(self, x, y):
        return  x + y + self.offset

if __name__=='__main__':
    calc = Calculator()
    print calc.add(2, 3) # 5
\end{minted}
\end{frame}
\begin{frame}
\frametitle{A word about \textbf{self}}
\label{sec-3-5}


\begin{itemize}
\item \texttt{self} is similar to \texttt{.this} in Java
\item Scope will be within a \emph{Class}
\end{itemize}
\end{frame}
\begin{frame}[fragile]
\frametitle{Module}
\label{sec-3-6}



\begin{minted}[]{python}
import Calculator

calc = Calculator()
calc.add(6, 7)  # 13
\end{minted}
\end{frame}
\begin{frame}[fragile]
\frametitle{Module}
\label{sec-3-7}



\begin{minted}[]{python}
from Calculator import add

add(6, 7)  # 13
\end{minted}
\end{frame}
\section{virtualenv}
\label{sec-4}
\begin{frame}[fragile]
\frametitle{Virtualenv}
\label{sec-4-1}


\begin{itemize}
\item Written in python
\end{itemize}

   \emph{Install - Ubuntu}


\begin{minted}[]{sh}
sudo apt-get install python-virtualenv
\end{minted}
\end{frame}
\begin{frame}[fragile]
\frametitle{Create a virtual environment}
\label{sec-4-2}
\begin{block}{Create}
\label{sec-4-2-1}


\begin{minted}[]{sh}
virtualenv ~/enigma
\end{minted}
\end{block}
\begin{block}{--no-site-packages}
\label{sec-4-2-2}

    \emph{Don't give access to global package directory to virtual     environment}

\begin{minted}[]{sh}
virtualenv --no-site-packages ~/enigma
\end{minted}
\end{block}
\end{frame}
\begin{frame}[fragile]
\frametitle{Activate/Deactivate}
\label{sec-4-3}
\begin{block}{Activate}
\label{sec-4-3-1}


\begin{minted}[]{sh}
source ~/enigma/bin/activate
\end{minted}
\end{block}
\begin{block}{Deactivate}
\label{sec-4-3-2}


\begin{minted}[]{sh}
deactivate
\end{minted}
\end{block}
\end{frame}
\begin{frame}[fragile]
\frametitle{pip}
\label{sec-4-4}
\begin{block}{Install packages}
\label{sec-4-4-1}


\begin{minted}[]{sh}
pip install pep8
pip install pylint
pip install django==1.5
\end{minted}
\end{block}
\begin{block}{List packages}
\label{sec-4-4-2}


\begin{minted}[]{sh}
pip list
pip freeze
\end{minted}
\end{block}
\end{frame}
\section{Reference}
\label{sec-5}
\begin{frame}
\frametitle{References}
\label{sec-5-1}

\begin{itemize}
\item Books
\begin{itemize}
\item Byte of Python
\item Dive into Python
\item Learn Python the Hard Way
\end{itemize}
\item Links
\begin{itemize}
\item \href{https://docs.python.org/2.7/tutorial/}{https://docs.python.org/2.7/tutorial/}
\item \href{https://docs.python.org/2/}{https://docs.python.org/2/}
\item \href{http://learnxinyminutes.com/docs/python/}{http://learnxinyminutes.com/docs/python/}
\end{itemize}
\end{itemize}
\end{frame}
\section{Contact}
\label{sec-6}
\begin{frame}
\frametitle{Contact}
\label{sec-6-1}

   Proudly made with Emacs org-mode and \LaTeX{}
\begin{block}{Contact}
\label{sec-6-1-1}

\begin{itemize}
\item \texttt{isachin@iitb.ac.in}
\item \href{https://github.com/psachin/slides/python}{https://github.com/psachin/slides/python}
\end{itemize}
\end{block}
\end{frame}

\end{document}
